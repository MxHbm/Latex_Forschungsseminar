\documentclass[final, english, english, a4paper, 12pt, numbers=noenddot, % Bei Änderung der Sprache 2 x kompilieren!
% tudscr spezifische Einstellungen
cd=false,
cdfont=false,cdfont=nohead,
cdmath=false,
cdhead=false,
cdfoot=true,
cdcover=monochrome,
cdgeometry=asymmetric,
declaration=heading,
declaration=notoc,
abstract=heading,
]{tudscrreprt}

%################ Hilfreiche Pakete laden #######################

%################ Hilfreiche Pakete laden #######################

\usepackage{settings/tudbwlimPackages}
\usepackage{settings/tudbwlimStyle}
\usepackage{scrhack}
\usepackage{microtype}
\usepackage{tikz,pgfplots}
\pgfplotsset{compat=newest}
\usepackage[chapter]{algorithm} % Falls das Paket floatrow geladen wird, muss dieses Paket danach geladen werden.
\iflanguage{english}{\floatname{algorithm}{Algorithm}\renewcommand{\listalgorithmname}{List of Algorithms}}{\floatname{algorithm}{Algorithmus}\renewcommand{\listalgorithmname}{Algorithmenverzeichnis}} % Algorithm-Umgebung an die verwendete Sprache anpassen
\usepackage{comment}
\usepackage{subcaption}
\usepackage{caption}
\usetikzlibrary{positioning}
\usetikzlibrary{mindmap}
\usepackage{graphicx}
\usepackage{float}
\usepackage{longtable}
\usepackage{xparse} % for \NewDocumentCommand
\usepackage{multirow}
\usepackage{array}
\usepackage{makecell}
\usepackage[justification=centering]{caption} % Für die Zentrierung der Bildunterschrift

%################ Notwendige Pakete laden #######################


\usepackage[bibencoding=auto,citestyle=authoryear-ibid,bibstyle=authoryear,maxcitenames=3,maxbibnames=10, backend = biber]{biblatex}% Vergessen Sie nicht in den Optionen das Bibliographieprogramm auf "biber" umzustellen! Um die Vorlage mit BibTeX nutzen zu können, muss die Option "backend=bibtex" übergeben werden. Es ist jedoch biber zu empfehlen, beachten Sie dazu die Hinweise der biblatex-Paketdokumentation im Abschnitt 3.15 "Using the fallback BibtTeX backend".
\usepackage{settings/BiblatexSetup}

\AfterPackage*{biblatex}%
{
    \RequirePackage[breaklinks=true, colorlinks=true, linktoc=section, linkcolor=blue, citecolor=black, hidelinks]{hyperref}
    % Da hyperref allerhand Veränderungen an vielen Standardbefehlen vornimmt, sollte dieses als letztes in der Präambel eingebunden werden. Nur Pakete, bei denen in der Dokumentation explizit darauf hingewiesen wird, dass diese nach hyperref zu laden sind, sollten auch danach folgen.
    \hypersetup{pdfprintscaling=None} % gleiches Verhalten, auch ohne hyperref, liefert: \pdfcatalog{/ViewerPreferences<</PrintScaling/None>>}
    \usepackage{footnote} % https://tex.stackexchange.com/questions/207192/footcite-in-float-caption
    \makesavenoteenv{figure}
    \makesavenoteenv{table}
    \makesavenoteenv{algorithm}
}


\AfterPackage*{hyperref}
{
    \RequirePackage[automake,acronym,symbols,nomain,translate=babel]{glossaries}
    \usepackage{settings/GlossariesSetup}
}
%############## Own commands #####################´
\newcommand{\parbreak}{\vspace{\baselineskip}\noindent}
\newcommand{\gendreauDataSet}{Gendreau (2006)}
\newcommand{\mouraDataSet}{Moura (2009)}
\newcommand{\ceschiaDataSet}{Ceschia (2013)}
\newcommand{\krebsADataSet}{Krebs (2021a)}
\newcommand{\krebsBDataSet}{Krebs (2021b)}
\newcommand{\gendreauDataSetText}{\gendreauDataSet\ }
\newcommand{\mouraDataSetText}{\mouraDataSet\ }
\newcommand{\ceschiaDataSetText}{\ceschiaDataSet\ }
\newcommand{\krebsADataSetText}{\krebsADataSet\ }
\newcommand{\krebsBDataSetText}{\krebsBDataSet\ }

%############## Ideas for packages #####################´
% Folgende auskommentierte Pakete sind als Vorschläge zu verstehen. Für Funktionsweise und Anwendungsfälle wird auf das Benutzerhandbuch "tudscr.pdf" (http://mirrors.ctan.org/macros/latex/contrib/tudscr/doc/tudscr.pdf) oder auf die entsprechende CTAN Dokumentation verwiesen.
%\usepackage{listings} %For Visualization of programming source code 
%\lstset{%
%	inputencoding=utf8,extendedchars=true,
%	literate=%
%	{ä}{{\"a}}1 {ö}{{\"o}}1 {ü}{{\"u}}1
%	{Ä}{{\"A}}1 {Ö}{{\"O}}1 {Ü}{{\"U}}1
%	{~}{{\textasciitilde}}1 {ß}{{\ss}}1
%	}

%\usepackage{calc} % Can be used to calculate inline distances in .tex files

%################ Sonstige Einstellungen/Befehle #####################
\onehalfspacing

%################ Abkürzungen #######################
\makeglossaries % Glossar, Abkürzungsverzeichnis und Symbolverzeichnis erstellen
\glsenableentrycount % aktiviert \cgls, \cglspl, \cGls, \cGlspl, siehe https://tex.stackexchange.com/questions/98494/glossaries-dont-print-single-occurences/230664#230664

% Abkürzungen, die im Abkürzungsverzeichnis auftauchen und automatisch durch das glossaries-Paket sortiert werden
\newacronym[description={Container Loading Problem}]{CLP}{CLP}{Container Loading Problem}
\newacronym[description={Vehicle Routing Problem}]{VRP}{VRP}{Vehicle Routing Problem}
\newacronym[description={Capacitated Vehicle Routing Problem}]{CVRP}{CVRP}{Capacitated Vehicle Routing Problem}
\newacronym[description={Bin Packing Problem}]{BPP}{BPP}{Bin Packing Problem}
\newacronym[description={Knapsack Problem}]{KP}{KP}{Knapsack Problem}
\newacronym[description={Last-In-First-Out}]{LIFO}{LIFO}{Last-In-–First-Out}
\newacronym[description={Cutting and Packing}]{CaP}{C\&P}{Cutting and Packing}
\newacronym[description={Two-dimensional}]{2D}{2D}{two-dimensional}
\newacronym[description={Three-dimensional}]{3D}{3D}{three-dimensional}
\newacronym[description={Constraint Programming}]{CP}{CP}{Constraint Programming}
\newacronym[description={Mixed-Integer Programming}]{MIP}{MIP}{Mixed-Integer Programming}
\newacronym[description={Genetic Algorithm}]{GA}{GA}{Genetic Algorithm}
\newacronym[description={Tabu Search}]{TS}{TS}{Tabu Search}
\newacronym[description={Simulated Annealing}]{SA}{SA}{Simulated Annealing}
\newacronym[description={Greedy Randomized Adaptive Search Procedure}]{GRASP}{GRASP}{Greedy Randomized Adaptive Search Procedure}
\newacronym[description={Variable Neighborhood Search}]{VNS}{VNS}{Variable Neighborhood Search}
\newacronym[description={Pallet Loading Problem}]{PLP}{PLP}{Pallet Loading Problem}
\newacronym[description={Bin Packing}]{BP}{BP}{Bin Packing}
\newacronym[description={Machine Learning}]{ML}{ML}{Machine Learning}
\newacronym[description = {Artificial Neural Network}]{ANN}{ANN}{Artificial Neural Network}
\newacronym[description = {Support Vector Machine}]{SVM}{SVM}{Support Vector Machine}
\newacronym[description = {Random Forest}]{RF}{RF}{Random Forest}
\newacronym[description = Two-dimensional Loading Capacitated Vehicle Routing Problem]{2L-CVRP}{2L-CVRP}{Two-dimensional Loading Capacitated Vehicle Routing Problem}
\newacronym[description = Three-dimensional Loading Capacitated Vehicle Routing Problem]{3L-CVRP}{3L-CVRP}{Three-dimensional Loading Capacitated Vehicle Routing Problem}
\newacronym[description = Three-Dimensional Loading Vehicle Routing Problem with Time Windows]{3L-VRPTW}{3L-VRPTW}{Three-Dimensional Loading Vehicle Routing Problem with Time Windows}


% Abkürzungen, die nicht im Abkürzungsverzeichnis aufgeführt werden
%\newabbreviation{zB}{z.\,B.}{zum Beispiel}

% Do i need this? 
\glsaddall[types=abbreviation]

% Symbole die im Symbolverzeichniss erscheinen sollen
% Mit "F5" kompilieren oder "Tools -> Befehle -> makeglossaries" (F9) starten, um das Symbolverzeichnis zu aktualisieren
% \newsymb{<sort by>}{<name>}{<symbol>}{<unit>}
%
\newsymb{E}{Erwartungswert}{\mathbb{E}\left(\cdot\right)}{}
\newsymb{P}{Wahrscheinlichkeitsmaß}{\mathbb{P}\left(\cdot\right)}{}
\newsymb{V}{Varianz}{\mathbb{V}\left(\cdot\right)}{}
\newsymb{X}{Zufallsvariable}{X}{}
%
% Oder so verwenden und im Fließtext dann mit $\ExpValue$ arbeiten:
%\newcommand{\ExpValue}{\mathbb{E}\left(\cdot\right)}
%\newsymb{E}{Erwartungswert}{\ExpValue}{}
%
\glsaddall[types={symbols}] % Alle Symbole werden dem Symbolverzeichnis hinzugefügt

%################ Bibliographie laden #######################
\addbibresource{./References.bib} % Pfad/Name der .bib-Datei

%################ Ende Präambel #######################
\pagenumbering{Roman}

\begin{document}
%################ Aufruf Deckblatt #######################
% mögliche Optionen, für weitere Informationen, siehe S. 23 des Benutzerhandbuchs des tudscr-Paket (http://mirrors.ctan.org/macros/latex/contrib/tudscr/doc/tudscr.pdf):

%%%%%%%%%%%%%%%%%%%%%%%%%%%%%%%%%%%%%
%\thesis{\diplomathesisname}		% Diplomarbeit/Diploma-Thesis
%\graduation[Dipl.-Kffr.]{Diplom-Kauffrau} % [Dipl.-Kfm.]{Diplom-Kaufmann}
%
%\thesis{\masterthesisname}			% Master-Arbeit/Master Thesis
%\graduation[M.Sc.]{Master of Science}
%
%\thesis{\bachelorthesisname}		% Bachelor-Arbeit/Bachelor Thesis
%\graduation[B.Sc.]{Bachelor of Science}

\renewcommand{\seminarCategoryText}{As part of the Industrial Management research seminar}
%
\newsavebox\thesisbox
\savebox\thesisbox{\normalfont\large\sffamily \parbox{\textwidth}{\vspace{6pt} \raggedright \seminarCategoryText}}
% https://github.com/tud-cd/tudscr/issues/85 %
%
\thesis{\seminarpapername \seminarCategory{\\\usebox\thesisbox}}		% Seminararbeit/Seminar Paper
%%%%%%%%%%%%%%%%%%%%%%%%%%%%%%%%%%%%%

\title{Feasibility prediction for the container loading problem}
%\subtitle{Optionaler Unter-/Zusatztitel}
\author{%
	Maximilian Hubmann
	\matriculationnumber{4805234}
	\dateofbirth{21.03.2000}
	\placeofbirth{Nürnberg}
	\course{Industrial Engineering and Management}
}
\date[]{\today}
\supervisor{Dipl.-Wi.-Ing. Florian Linß}
\professor{Prof. Dr. Udo Buscher}

\setcounter{page}{1}

%%% IM 
\headlogo{./pictures/IM-Logo}
\faculty{Faculty of business and economics}
\chair{Chair of Business Administration, in particular Industrial Management}
\maketitle[cdtitle=true, cdfont=false]

%################ Abstract ######################
%\begin{abstract}
%Das ist ja maal so cool, dass ich ich einfahc hier einen Tsext schreiben kann
%\end{abstract}

%%% Danksagung
%\danke{Danksagung}{Text}

%####################### Verzeichnisse ################%
%\microtypesetup{protrusion=false}
% Inhaltsverzeichnis
\tableofcontents

% Abbildungsverzeichnis (falls nichts benötigt, einfach als Kommentar setzen)
%\listoffigures
%\addcontentsline{toc}{chapter}{\listfigurename}

% Tabellenverzeichnis (falls nichts benötigt, einfach als Kommentar setzen)
%\listoftables
%\addcontentsline{toc}{chapter}{\listtablename}

% Algorithmenverzeichnis (falls nichts benötigt, einfach als Kommentar setzen)
%\listofalgorithms
%\addcontentsline{toc}{chapter}{\listalgorithmname}

\microtypesetup{protrusion=true}

%Abkürzungsverzeichnis (falls nichts benötigt, einfach als Blockkommentar setzen)
%\printacronyms[style=bwlimsuper]
%\addcontentsline{toc}{chapter}{\acronymname}

% Symbolverzeichnis (falls nichts benötigt, einfach als Blockkommentar setzen)
%\printsymbols[style=symblong, title=\listsymbolname]
%\addcontentsline{toc}{chapter}{\listsymbolname}


%############### Einstellungen für Fließtext setzen ####################
\clearpage
\setcounter{page}{1}\pagenumbering{arabic}


%####################### Fließtext ####################

% Introduction
\chapter{Erstes Kapitel}

Das \cgls{scm} eines \glsuseri{kmu} ist auszubauen. \cGls{mw} ist groß. \cGls{mw} sollte \acrshort{zB} am Anfang eines Satzes groß geschrieben werden. \glqq \Acrlong{zB}\grqq\xspace kann auch ausgeschrieben werden. Die Befehle \verb|\cgls{}| bzw. \verb|\cGls{}| zählen die Verwendung einer Abkürzung und fügen sie dem Abkürzungsverzeichnis hinzu, wenn sie mindestens zweimal verwendet wird. Text Text Text Text Text Text Text Text Text Text Text Text Text Text Text Text Text Text Text Text Text Text Text Text Text Text Text Text Text Text Text Text Text Text Text Text Text Text Text Text Text Text Text Text Text Text Text Text Text Text \glssymbol{E} Text \glssymbol{V} Text \glssymbol{P} Text \glssymbol{X} Text Text Text Text Text Text Text Text Text Text Text Text.

Text \cgls{crm} Text Text Text Text Text Text Text Text Text Text Text Text Text Text Text Text Text Text Text Text Text Text Text Text Text Text Text Text Text Text Text Text Text Text Text Text Text Text Text Text Text Text Text Text Text Text Text Text Text Text Text Text Text Text Text Text Text Text Text Text Text Text Text Text Text Text Text Text Text Text Text Text Text Text Text Text Text Text Text Text Text Text Text Text Text Text Text Text Text Text Text Text Text Text Text Text Text Abbildung~\ref{fig:1}.%

\begin{figure}[ht]
    \centering
    \includegraphics[width=0.1\textwidth]{./pictures/IM-Logo}
    \caption[Erste Abbildung]{Erste Abbildung\footnotemark}\label{fig:1}
\end{figure}\footnotetext{Eigene Abbildung.}

Text Text Text Text Test Test Text Text Text Text Text Text Text Text Text Text Text Text Text Text Text Text Text Text Text Text Text Text Text Text Text Text Text Text Text Text Text Text Text Text Text Text Text Text Text Text Text Text Text Text Text Text Text Text Text Text Text Text Text Text Text Text Text Text Text Text Text Text Text Text Text Text Text Text Text Text Text Text Text Text Text Text Text Text Text Text Text Text Text Text Text Text Text Text Text Text Text Text Text Text Text Text Text Text Text Text Text Text Text Text Text \cgls{scm} Text Text Text Text Text Text Text Text Text Text Text Text Text Text Text Text Text Text Text Text Text Text Text Text Text Text Text Text Text Text Text Text Text Text Text Text Text Text Text Text Text Text Text Text Text Text Text Text Text Text Text Text Text Text Text Text Text Text Text Text Text Text Text Text\footnote{Fußnotentext.} Text Text Text Text Text Text Text Text Tabelle~\ref{tab:1}.%


\begin{table}[tb]
    \centering
    \begin{tabular}{ccc}
        \toprule
        A & B & C \\
        \midrule
        1 & 2 & 3 \\
        4 & 5 & 6 \\
        \bottomrule
    \end{tabular}
    \caption{Erste Tabelle}\label{tab:1}
\end{table}
Text Text Text Text Text Text Text Text Text Text Text Text Text Text Text Text Text Text Text Text Text Text Text Text Text Text Text Text Text Text Text Text Text Text Text Text Text Text Text Text Text Text Text \cgls{kmu} Text Text Text Text Text Text Text Text Text Text Text Text Text Text Text Text Text Text Text Text Text Text Text Text Text Text Text Text Text Text Text Text Text Text Text Text Text Text Text Text Text Text Text Text Text Text Text Text Text Text Text Text. Wie \citeauthor{bortfeldt_constraints_2013} zeigen, Text Text.\footcite[Vgl.][33]{ceschia_local_2013} Text Text \cgls{crm}.

\begin{figure}[h]
    \centering
    \begin{tikzpicture}%[scale=0.975]
        \begin{axis}[
                xmin=-0.5, xmax=5,
                ymin=-0.5, ymax=5,
                xtick={1,...,4},
                ytick={1,...,4},
                axis y line=center,
                axis x line=middle,
                ylabel={$y$},
                xlabel={$x$},
                xlabel style={below right},
                ylabel style={above left},
            ]

            \node (A) at (axis cs:2,1) {};
            \node[right] at (A) {$A$};
            \draw[fill] (A) circle [radius=1.75pt];

        \end{axis}
    \end{tikzpicture}
    \caption[Abbildungsunterschrift]{Abbildungsunterschrift\footcite[Vgl.][39]{zhang_hybrid_2017}}
\end{figure}


Text Text Text Text Text Text Text Text Text Text Text Text Text Text Text Text Text Text Text Text Text Text Text Text Text Text Text Text Text Text Text Text Text Text Text Text Text Text Text Text Text Text Text Text Text Text Text Text Text Text Text Text Text Text Text Text Text Text Text Text Text Text Text Text Text Text Text Text Text Text Text Text Text Text Text Text Text Text Text Text Text Text Text Text Text Text Text Text Text Text Text Text Text. Verweis mit mehreren Quellenangaben.\footcites(Vgl.)()[][24]{gendreau_tabu_2006}[][15]{krebs_advanced_2021}


\chapter{The Container Loading Problem}
\section{Formal Problem Definition}
\section{Computational Complexity and NP-hardness}
\section{Constraints in the CLP}
\section{Related Problems: Bin Packing, Knapsack, Cutting Stock}

The placement and assignment of three-dimensional items to larger, mostly rectangular
containers, is a well known problem in logistics and operations research, as the
potential of cost savings and efficiency gains is substantial by reducing the number
of needed containers or by fulfilling customer needs. These problems are differentiated
by \cite{bortfeldt_constraints_2013} in \textit{input minimization}, where the number
of needed containers is minimized, and \textit{output maximization} problems, where the
value of the associated items is maximized.

Differentiate the Multiple Container Loading Problem, from Knapsack problem,
and cutting stock problem \parencite{bortfeldt_constraints_2013}

Use systemization of \parencite{bortfeldt_constraints_2013} to group problems in
groups

Define Items, and how they are meant here. Differnetiate between heterogenous
and homogenous items.

Define the container.

\begin{figure}[H]
    \centering
    \includegraphics[width=7cm]{pictures/3l_cvrp_example.png}
    \caption{Visualization from \cite{tamke_branch-and-cut_2024}.}
    \label{fig:solution-visualization}
\end{figure}
% Main Part

\chapter{Classical Solution Approaches}
\section{Exact Algorithms}
\section{Heuristic and Metaheuristic Methods}
\section{Strengths and Limitations of Classical Approaches}

\chapter{Motivation for Feasibility Prediction}
\section{Advantages of Predicting Feasibility}
\section{Trade-off Between Computation Time and Decision Support}
\section{Use Cases: Pre-filtering and Early-stage Decisions}
\section{Feasibility Prediction: Methods and Approaches}
\section{Overview of Machine Learning Classifiers}
\section{Data Dependency and Bias}

\chapter{Comparison of Available Datasets}
\section{Overview of Public Datasets}
\section{Evaluation Criteria: Diversity, Realism, Labeling, Size}
\section{Comparison of Selected Datasets}
\section{Justification for Dataset Selection}

\chapter{Conclusion and Outlook}
\section{Summary of Key Insights}
\section{Practical Implications}
\section{Future Work and Research Directions}



%####################### Appendix #########################
%%############################# Anhang #################################
%\appendix
%\chapter{Mathematischer Anhang}
%\chapter{Programmcodes}

\clearpage
%####################### Literaturverzeichnis #########################
\addcontentsline{toc}{chapter}{\bibname}
\printbibliography
%####################### Selbständigkeitserklärung ####################

\newConfirmationEnglish{Dresden}

%####################### Platzhalter - Abkürzungsverzeichnis muss erstellt werden####################

\printacronyms[style=bwlimsuper]

%####################### End of Document ####################
\end{document}