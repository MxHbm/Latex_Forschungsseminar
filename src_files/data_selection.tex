\chapter{Comparison of Available Datasets}
\label{sec:dataset_selection}

This chapter builds upon the insights elaborated in previous chapters to lay the foundation
for developing a classifier for the \cgls{3L-CVRP} with practical constraints by comparing and
selecting a suitable dataset. The classifier will be used to predict the feasibility of packing a
given set of cuboids into a vehicle - representing a single tour - similar to the approach presented
in Chapter~\ref{sec:motivation_feasibility_prediction} by \citeauthor{zhang_learning-based_2022}.

As discussed in Chapter~\ref{sec:classical_solution_approaches}, the \gls{CLP} is an NP-hard problem,
making it computationally demanding to verify the packing feasibility of individual routes.
A classifier can therefore enhance the performance of existing exact algorithms \footcite[cf.][]{tamke_branch-and-cut_2024}
by reducing the need to compute feasibility exactly - using \cgls{CP} methods - only when a
final solution is found, or just before columns generated in Branch-and-Price or Branch-and-Cut
algorithms are discarded \footcite[cf.][pp. 9--11]{zhang_learning-based_2022}. As many single routes
need to be evaluated this use case has the potential to significantly reduce the overall
computation time. First, five published \cgls{3L-CVRP} datasets will be compared with respect
to their overall characteristics. Subsequently, a more detailed analysis will focus on the
heterogeneity of the items. Finally, a suitable dataset will be selected as a promising foundation
for developing a classifier.

The first \cgls{3L-CVRP} dataset was published by \citeauthor*{gendreau_tabu_2006} in
\citeyear{gendreau_tabu_2006} and delivered the first \cgls{3L-CVRP} instances.
The second dataset was published by \citeauthor*{moura_integrated_2009} in \citeyear{moura_integrated_2009},
and combines the Solomon benchmark instances\footcite[cf.][]{solomon_algorithms_1987} and
the \gls{CLP} instances from \citeauthor*{bischoff_issues_1995} defining the \gls{3L-VRPTW}.
\citeauthor*{ceschia_local_2013} published a \cgls{3L-CVRP} dataset in \citeyear{ceschia_local_2013}
representing real-life data and \citeauthor*{krebs_advanced_2021} published two different datasets
in \citeyear{krebs_advanced_2021} with a focus on more realistic constraints. The characteristics of the
single different datasets are summarized in the follwing Table~\ref{tab:dataset_comparison}. It needs
to be noted, that all instances consider either the \cgls{3L-CVRP} or the \cgls{3L-VRPTW} and the considered
constraints are not identical, which are shown in Table~\ref{tab:constraints_cvrp_instances}.

\clearpage
\begin{table}[h]
    \centering
    \small
    \begin{tabular}{@{}lcccc@{}}
        \toprule
        \textbf{Reference}                                  & \textbf{Instances} & \textbf{Customers}  & \textbf{Rel. Mass Range} & \textbf{Orders Range}   \\
        \midrule
        Gendreau, 2006\footcite[cf.][]{gendreau_tabu_2006}  & 27                 & [15, 100]           & [0, 0.91]                & [1, 3]                  \\
        Moura, 2009\footcite[cf.][]{moura_integrated_2009}  & 46                 & 25                  & [0, 0.01]                & [30, 100]               \\
        Ceschia, 2013\footcite[cf.][]{ceschia_local_2013}   & 13                 & [11, 129]           & [0, 0.05]                & [1, 41]                 \\
        Krebs\_a, 2021\footcite[cf.][]{krebs_advanced_2021} & 600                & 20, 60, 100         & [0, 0.58]                & [1, 30]                 \\
        Krebs\_b, 2021\footcite[cf.][]{krebs_axle_2021}     & 80                 & 30, 60, 90, 120     & [0, 0.04]                & [1, 22]                 \\
        \toprule
        \textbf{Reference}                                  & \textbf{Items}     & \textbf{Item Types} & \textbf{Min dimension}   & \textbf{Max dimensions} \\
        \midrule
        Gendreau, 2006                                      & [26, 199]          & [26, 199]           & \{12, 5, 6\}             & \{36, 15, 18\}          \\
        Moura, 2009                                         & 1050, 1550         & 5                   & \{8.1, 3.3, 2.5\}        & \{12, 9.9, 7.3\}        \\
        Ceschia, 2013                                       & [254, 8060]        & [9, 97]             & \{0.1, 0.5, 0.1\}        & \{66, 23, 28.4\}        \\
        Krebs\_a, 2021                                      & 200, 400           & 3, 10, 100          & \{6, 2, 2\}              & \{35, 14, 14\}          \\
        Krebs\_b, 2021                                      & 200, 400           & 10, 100             & \{6, 6, 6\}              & \{25, 12, 15\}          \\
        \bottomrule
    \end{tabular}
    \caption{Numeric comparisons between available datasets.}
    \label{tab:dataset_comparison}
\end{table}
\begin{comment}
\begin{table}[!ht]
    \centering
    \small
    \begin{tabular}{@{}lcccc@{}}
        \toprule
        \textbf{Reference}                                  & \textbf{Instances} & \textbf{Customers}  & \textbf{Rel. Mass Range} & \textbf{Orders Range}   \\
        \midrule
        Gendreau, 2006\footcite[cf.][]{gendreau_tabu_2006}  & 27                 & [15, 100]           & [0, 0.91]                & [1, 3]                  \\
        Moura, 2009\footcite[cf.][]{moura_integrated_2009}  & 46                 & 25                  & [0, 0.01]                & [30, 100]               \\
        Ceschia, 2013\footcite[cf.][]{ceschia_local_2013}   & 13                 & [11, 129]           & [0, 0.05]                & [1, 41]                 \\
        Krebs\_a, 2021\footcite[cf.][]{krebs_advanced_2021} & 600                & \{20, 60, 100\}     & [0, 0.58]                & [1, 30]                 \\
        Krebs\_b, 2021\footcite[cf.][]{krebs_axle_2021}     & 80                 & \{30, 60, 90, 120\} & [0, 0.04]                & [1, 22]                 \\
        \toprule
        \textbf{Reference}                                  & \textbf{Items}     & \textbf{Item Types} & \textbf{Min dimension}   & \textbf{Max dimensions} \\
        \midrule
        Gendreau, 2006                                      & [26, 199]          & [26, 199]           & \{12, 5, 6\}             & \{36, 15, 18\}          \\
        Moura, 2009                                         & \{1050, 1550\}     & 5                   & \{8.1, 3.3, 2.5\}        & \{12, 9.9, 7.3\}        \\
        Ceschia, 2013                                       & [254, 8060]        & [9, 97]             & \{0.1, 0.5, 0.1\}        & \{66, 23, 28.4\}        \\
        Krebs\_a, 2021                                      & \{200, 400\}       & \{3, 10, 100\}      & \{6, 2, 2\}              & \{35, 14, 14\}          \\
        Krebs\_b, 2021                                      & \{200, 400\}       & \{10, 100\}         & \{6, 6, 6\}              & \{25, 12, 15\}          \\
        \bottomrule
    \end{tabular}
    \caption{Numeric comparisons between available datasets.}
    \label{tab:dataset_comparison}
\end{table}
\end{comment}

The brackets [\,] indicate a range of possible values, whereas $\{\,\}$ symbolizes tuples and all integers
single values. The table summarizes the numerical characteristics of the datasets and differences
can easily be identified. It is important to note, that the relative mass of the datasets differs
significantly. Whereas Gendreau and Krebs\_a, have the widest range. Relative is defined as in
comparison to the vehicle capacity and dimensions, as the datasets are not directly comparable.
Furtermore, differs the numbers of ordered items quite heavenly, whereas Gendreau
considers max 3 requestes items per customer, it can be up to 100 in Moura.
The total requested items differs also quite heavely between the dataset, as well as the different item types. A item type is defined
by its geometrical dimensions, the weight, and possible stability characteristics, such as fragility.


\newcolumntype{L}[1]{>{\raggedright\arraybackslash}p{#1}} % left-aligned
\newcolumntype{C}[1]{>{\centering\arraybackslash}p{#1}}   % centered
\begin{table}[ht]
    \centering
    \small
    \renewcommand{\arraystretch}{1.2}
    \setlength{\tabcolsep}{6pt}
    \begin{tabular}{@{}L{3cm} L{3cm} C{1.2cm} C{1.2cm} C{1.2cm} C{1.2cm} C{1.2cm}@{}}
        \toprule
        \textbf{Category}                   & \textbf{Constraint} & \rotatebox{90}{Gendreau, 2006} & \rotatebox{90}{Moura, 2009} & \rotatebox{90}{Ceschia, 2013} & \rotatebox{90}{Krebs, 2021a} & \rotatebox{90}{Krebs, 2021b} \\
        \midrule
        \multirow{2}{*}{Orientation}        & Full Rotation       &                                & $\bullet$                   &                               &                              &                              \\
                                            & Vertical Rotation   & $\bullet$                      &                             & $\bullet$                     & $\bullet$                    & $\bullet$                    \\\midrule
        \multirow{3}{*}{Vertical Stability} & Fragility           & $\bullet$                      &                             & $\bullet$                     & $\bullet$                    & $\bullet$                    \\
                                            & Support Area        & $\bullet$                      & $\bullet$                   & $\bullet$                     & $\bullet$                    & $\bullet$                    \\
                                            & LB Strength         &                                &                             &                               & $\bullet$                    &                              \\\midrule
        \multirow{3}{*}{Stability}          & Balanced Loading    &                                &                             & $\bullet$                     & $\bullet$                    &                              \\
                                            & Axle Weights        &                                &                             &                               & $\bullet$                    & $\bullet$                    \\
                                            & Robust Stability    &                                &                             & $\bullet$                     & $\bullet$                    &                              \\\midrule
        \multirow{5}{*}{Unloading}          & Reachability        &                                &                             & $\bullet$                     & $\bullet$                    &                              \\
                                            & Sequence            & $\bullet$                      &                             &                               & $\bullet$                    &                              \\
                                            & Time Windows        &                                & $\bullet$                   &                               & $\bullet$                    &                              \\
                                            & LIFO                & $\bullet$                      & $\bullet$                   & $\bullet$                     & $\bullet$                    & $\bullet$                    \\
                                            & MLIFO               &                                &                             & $\bullet$                     & $\bullet$                    &                              \\
        \bottomrule
    \end{tabular}
    \caption{Matrix overview of constraints covered in selected CVRP datasets. A bullet ($\bullet$) indicates that the constraint is considered in the respective dataset.}
    \label{tab:constraint_matrix}
\end{table}
The second table gives an overview of the considered constraints. All the basic constraints are considered,
and further constraints, which are data set related, which are defined in Chapter~\ref{sec:clp_definition}.

As the comparison and selection of the datasets is quite difficult, characteristics of the data are are plotted
to visualize other characteristics in Figure~\ref{fig:dataset_comparison}.
\begin{figure}[ht]
    \centering
    \includegraphics[width=0.8\textwidth]{pictures/comparison_datasets_3lcvrp.png}
    \caption{Visualized 3D packing with packing constraints}.
    \label{fig:dataset_comparison}
\end{figure}

Hier schreiben was wichitg ist bei der auswahl für classifier, viel fälle abdecken an instanzen,
unterschiedlich groß, undterschiedlich viele items mit ganz unterschiedlichen item anforderungen
und bedarfen.

\begin{figure}[ht]
    \centering
    \includegraphics[width=0.8\textwidth]{pictures/volume_over_instances_krebs.png}
    \caption{Visualized 3D packing with packing constraints}.
    \label{fig:krebs_dataset_analysis}
\end{figure}


%\section{Overview of Public Datasets}
%\section{Evaluation Criteria: Diversity, Realism, Labeling, Size}
%\section{Comparison of Selected Datasets}
%\section{Justification for Dataset Selection}