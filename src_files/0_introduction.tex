\chapter{Introduction}
\label{sec:introduction}
Transporting goods has become a major task in a globalized world, with parcel shipping alone covering
$161$ billion parcels in 2022, resulting in over $5000$ parcels delivered per second \footcite[cf.][]{noauthor_parcel_nodate}.
These staggering numbers are only manageable through a strong logistics network that efficiently ships
items from companies to single customers, meeting their needs. Beyond parcels, maritime transport
handles $80\%$ of global trade, where uniform containers are packed with differently shaped cargo with
individual requirements \footcite[cf.][]{united_nations_trade_and_development_review_2024}.
A central challenge faced in this domain is the
efficient arrangement of items of various shapes and sizes. Given the vast scale of logistics, even small inefficiencies become costly. Thus, there is strong
interest in optimizing space and minimizing container use.
This \gls{CLP} has  broad range of applications occuring in many real-world scenarios,
such as loading cargo, as well as optimizing warehouse storage and
packing of smaller goods into pallets or cardboard boxes\footcite[cf.][p.1]{bortfeldt_constraints_2013}.
Even though, cargo is transported one a huge scale by maritime super ships. There is a huge need of transporting
single items and delivieries with trucks to the demanding companies and individuals for the last mile,
covering the urge to find best possible routes to load vehicles efficiently and including as many as customers
as possible. The task for finding cost efficient route with integrated loading is called is a special interesta
area of research, developing efficient heuristics for finding practical good solutions. However, especially
in the field of \gls{3L-CVRP}, exact solution methods are scarcly found due to the NP-hardness of the problem
and the complex  computationally expensive, especially for large, constraint-rich instances
involving orientation rules, stability, and load-bearing limits, which are crucial for practical and safe
operations\footcite[cf.][p.377--378]{bischoff_issues_1995}. Therefore aims this paper to provide an overview
of the \gls{CLP} and explore whether \gls{ML} techniques could support or enhance classical exact solution methods
by leveraging the speed and the possibiltiy not to check every route.
Chapter~\ref{sec:clp_definition} introduces the \gls{CLP} and its common constraints.
Chapter~\ref{sec:classical_solution_approaches} outlines exact and heuristic algorithms.
Chapter~\ref{sec:motivation_feasibility_prediction} presents recent research on integrating \gls{ML}
models to improve solution quality and speed. Chapter~\ref{sec:dataset_selection} compares datasets
suitable for training classifiers for the constrained \gls{3L-CVRP}. Finally, Chapter~\ref{sec:conclusion}
summarizes insights and discusses future work in realistic feasibility prediction for the \gls{3L-CVRP}.


\begin{comment}
First, the \gls{CLP} and the most common
constraints faced in practical scenarios are presented in Chapter~\ref{sec:clp_definition}. Afterwards,
exact and heuristic solution algorithms are presented in Chapter~\ref{sec:classical_solution_approaches}.
Then, recent research about integrating gls{ML} models in \gls{CLP} algorithms is presented in Chapter~\ref{sec:motivation_feasibility_prediction} and
their potential to improve solution quality and computation time rapidly. Based on these findings several
datasets are compared to each other in Chapter~\ref{sec:dataset_selection} to identify a suitable dataset
for training a classifier for the \gls{3L-CVRP} with constraints. Finally, this paper aims to bridge this gap by examining constraint
datasets relevant to the \gls{3L-CVRP}, thereby enabling more realistic feasibility predictions for
future work, which is depicted in Chapter~\ref{sec:conclusion}.
\end{comment}
