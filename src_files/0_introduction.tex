\chapter{Introduction}
\label{sec:introduction}
Efficiently arranging items of varying shapes and sizes within a container is a fundamental challenge
in logistics and supply chain management. This \gls{CLP} arises in numerous real-world contexts,
such as loading cargo onto ships, trucks, and aircraft, as well as optimizing warehouse storage and
packing of smaller goods into pallets or cardboard boxes\footcite[cf.][p.1]{bortfeldt_constraints_2013}.

\parbreak

The \gls{CLP}, particularly in its multi-dimensional variants, is classified as an NP-hard problem.
Real-world constraints—such as orientation rules, stability requirements, and load-bearing
limitations—add further complexity to the problem\footcite[cf.][p.377-378]{bischoff_issues_1995},
and are elaborated in Chapter~\ref{sec:clp_definition}. As such, exact algorithms are limited to
small instances, and research has focused on heuristic approaches,
as discussed in Chapter~\ref{sec:classical_solution_approaches}.

\parbreak

In complex scenarios, recent research demonstrates that \gls{ML} models can help to reduce the computation
time finding good solutions. Especially classifiers were succesfully integrated in existing algorithms
and are presenred in Chapter~\ref{sec:motivation_feasibility_prediction} \footcite[cf.][]{zhang_learning-based_2022}.
A suitable use case for using a classifier is the \gls{3L-CVRP}, where single tours of the \gls{VRP}
represent a \gls{CLP} with cuboids themselves and numerous tours need to be comutational expensively checked for feasibility
\footcite[cf.][p. 1]{tamke_branch-and-cut_2024}. However, current research of classifiers of tours
rely on simplified assumptions that do not fully capture practical constraints. Therefore, suitable \gls{3L-CVRP}
datasets are compared in Chapter~\ref{sec:dataset_selection} and a suitable dataset is selected to
lay the foundation for training a classifier. Finally, this paper aims to bridge this gap by examining constraint
datasets relevant to the \gls{3L-CVRP}, thereby enabling more realistic feasibility predictions for
future work, which is depicted in Chapter~\ref{sec:conclusion}.

