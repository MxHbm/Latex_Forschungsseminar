\chapter{Classical Solution Approaches}
The \gls{CLP} with muli-dimensions is a NP-hard combinatorial problem \footcite[cf.][p.11]{bortfeldt_constraints_2013}.
As a consequence, heuristics and metaheuristics were the dominating tools
in the early stages of this research field  \footcite[cf.][]{pisinger_heuristics_2002}.
However, the number of papers studying exact solution methods and their solution quality
for the \gls{2D} \gls{CLP} has significantly increased in the recent years \footcite[cf.][p.23]{iori_exact_2021}.
In the remainder of this seminar, the focus will be restricted to the three-dimensional \gls{CLP},
which poses additional complexity and remains less explored in the literature.
\citeauthor{zhao_comparative_2016} conducted a comprehensive review of the solution methods
for the \gls{3D} \gls{CLP} based on the overview from \citeauthor{bortfeldt_constraints_2013}.
The following figure \ref{fig:solution_methods_overview} presents a limited overview of
selected solution methods, divided in three categories constructive heuristics, metaheuristics
and exact methods. A fourth category, hybrid methods, is not explicitly shown in the figure, as it combines different
components from the other three categories.

\begin{figure}[htbp]
    \centering
    \resizebox{0.55\linewidth}{!}
    {
        \begin{tikzpicture}[grow cyclic, text width=2.7cm, align=flush center,
                level 1/.style={level distance=3.5cm,sibling angle=120},
                level 2/.style={level distance=3.5cm,sibling angle=55}
            ]

            \node {Solution Techniques \gls{CLP}}
            child { node {Exact \\ Methods}
                    child { node {\Gls{CP}}}
                    child { node {\Gls{MIP}}}
                    child { node {Branch-and-\\Cut}}
                    child { node {Branch-and-\\Price}}
                }
            child { node {Metaheuristics}
                    child { node[text width=3cm] {\Gls{GRASP}}}
                    child { node {\Gls{SA}}}
                    child { node {\Gls{TS}}}
                    child { node {\Gls{GA}}}
                }
            child { node {Constructive Heuristics}
                    child { node {Wall Building}}
                    child { node {Layer Building}}
                    child { node {Block based packing}}
                    child { node {Tower based packing}}
                }
            ;

        \end{tikzpicture}
    }
    \caption[Overview of solution methods for the CLP.]{Overview of solution methods for the CLP.}
    \label{fig:solution_methods_overview}
\end{figure}



These categories will now be explained more in detail.

\subsubsection{Constructive Heuristics}

Depending on the heterogeneity of the items, two primary strategies are commonly used to
obtain an initial feasible solution. In cases where the items are weakly homogeneous,
the problem's dimensionality is reduced from \gls{3D} to \gls{2D} by constructing either
\textbf{walls} or \textbf{layers} of similarly sized items. These structures fill one
dimension of the container—typically either length and height or width and height—thereby
simplifying the remaining problem space. Moreover, when these layers or walls are of
approximately equal dimensions, both horizontal and vertical stability constraints are
inherently satisfied.
Beyond these two dominant approaches, several adaptations exist to address item heterogeneity.
For example, \citeauthor{gehring1997genetic} propose the construction of item
\textbf{towers}, in which boxes are stacked on top of others with identical base dimensions.
This effectively reduces the 3D problem to a 2D floor space arrangement,
similar to pallet-loading scenarios. Another method involves forming \textbf{blocks}
composed of identically shaped items. These blocks are treated as single entities
during packing, significantly reducing the number of elements to be handled and thus
lowering overall problem complexity.


\subsubsection{Constructive Heuristics}
Depending on the heterogenity of the items, two possible approaches for finding the first
feasible solution exist. When the items are weakly homogenous, the algorithms aim to reduce
the the complexity from \gls{3D} to \gls{2D} by \textbf{building walls or layers} of homogenous items.
These layers fill one dimension of the container, whether length and width or length and width.
Consequently, the remaining space of the container is reduced and the problem becomes smaller.
Additionally, when these layers and walls have approcimately the same dimensions, horizontal
and vertical stability constraints are met. Apart from these two dominant/ well known approaches,
adaptions exist. For heterogeneous items, \citeauthor{gehring1997genetic},
suggest to build towers of items, stacking, them on the exact identical sized area, and then
concentrate only of floor space of each tower, similar to pallet loading problems. Another approach
is the building of blocks of idetnicla shapoed items, to reduce the number of boxes and handle
only the to blocks constructed item, which reduces the complexity of the problem.

As the variety of \gls{CLP}s is quite broad, this work focuses specifically on
the application to \gls{VRP} with integrated \gls{3D} container loading. The emphasis lies
on the verification of packing feasibility, as the total distance is minimized.

\subsubsection{Heuristics and Metaheuristics}
\subsubsection{Exact Algorithms}