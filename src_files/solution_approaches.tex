\chapter{Classical Solution Approaches}
\label{sec:classical_solution_approaches}
The \gls{CLP} with muli-dimensions is a NP-hard combinatorial problem \footcite[cf.][p.11]{bortfeldt_constraints_2013}.
As a consequence, heuristics and metaheuristics were the dominating tools
in the early stages of this research field  \footcite[cf.][]{pisinger_heuristics_2002}.
However, the number of papers studying exact solution methods and their solution quality
for the \gls{2D} \gls{CLP} has significantly increased in the recent years \footcite[cf.][p.23]{iori_exact_2021}.
In the remainder of this seminar, the focus will be restricted to the three-dimensional \gls{CLP},
which poses additional complexity and remains less explored in the literature.
\citeauthor{zhao_comparative_2016} conducted a comprehensive review of the solution methods
for the \gls{3D} \gls{CLP} based on the overview from \citeauthor{bortfeldt_constraints_2013}.
The following figure \ref{fig:solution_methods_overview} presents a limited overview of
selected solution methods, divided in three categories constructive heuristics, metaheuristics
and exact methods. A fourth category, hybrid methods, is not explicitly shown in the figure, as it combines different
components from the other three categories.

\begin{figure}[htbp]
    \centering
    \resizebox{0.55\linewidth}{!}
    {
        \begin{tikzpicture}[grow cyclic, text width=2.7cm, align=flush center,
                level 1/.style={level distance=3.5cm,sibling angle=120},
                level 2/.style={level distance=3.5cm,sibling angle=55}
            ]

            \node {Solution Techniques \gls{CLP}}
            child { node {Exact \\ Methods}
                    child { node {\Gls{CP}}}
                    child { node {\Gls{MIP}}}
                    child { node {Branch-and-\\Cut}}
                    child { node {Branch-and-\\Price}}
                }
            child { node {Metaheuristics}
                    child { node[text width=3cm] {\Gls{GRASP}}}
                    child { node {\Gls{SA}}}
                    child { node {\Gls{TS}}}
                    child { node {\Gls{GA}}}
                }
            child { node {Constructive Heuristics}
                    child { node {Wall Building}}
                    child { node {Layer Building}}
                    child { node {Block based packing}}
                    child { node {Tower based packing}}
                }
            ;

        \end{tikzpicture}
    }
    \caption[Overview of solution methods for the CLP.]{Overview of solution methods for the CLP.}
    \label{fig:solution_methods_overview}
\end{figure}



These categories will now be explained more in detail.

\subsubsection{Constructive Heuristics}
Depending on the heterogeneity of the items, two primary strategies are commonly used to
obtain an initial feasible solution. In cases where the items are weakly homogeneous,
the problem's dimensionality is reduced from \gls{3D} to \gls{2D} by constructing either
\textbf{walls} or \textbf{layers} of similarly sized items. These structures fill one
dimension of the container—typically either length and height or width and length, thereby
simplifying the remaining problem space. Moreover, when these layers or walls are of
approximately equal dimensions, both horizontal and vertical stability constraints are met.
Beyond these two dominant approaches, several adaptations exist to address item heterogeneity.
For example, \citeauthor{gehring_genetic_1997} propose the construction of item
\textbf{towers}, in which boxes are stacked on top of others with identical base dimensions.
This effectively reduces the 3D problem to a 2D floor space arrangement,
similar to pallet-loading scenarios. Another method involves forming \textbf{blocks}
composed of identically shaped items. These blocks are treated as single entities
during packing, significantly reducing the number of elements to be handled and thus
lowering overall problem complexity.

\subsubsection{Metaheuristics}
Once an initial solution is found, metaheuristics or improvement heuristics can be applied
to enhance it. In contrast to constructive heuristics, these methods require a solution
representation such as a permutation of items in order to perform item moves. \gls{GA}s were
used to either improve the walls, towers or layers constructed by the constructive heuristics
or by improving the quality of the permutation of all items.
\gls{TS} in general is widely spread approach for \gls{BPP} and \gls{CLP} problems. It is based on
the idea of iteratively improving a solution by exploring its neighborhood and simultaneously
storing certain moves or complete solutions in a tabu list to avoid cycling back to
previous solutions, feasible or infeasible ones. \gls{SA} has rarely been used as a
standalone metaheuristic in the context of \gls{3D} \gls{CLP}. Instead, it is often combine
with other approaches to leverage its cooling schedule, which allows the acceptance of
worse solutions at higher temperatures. This helps the algorithm to escape local optima
early in the search and gradually transition into a more focused intensification phase as
the temperature decreases. \gls{GRASP} has the advantage of controlling the selection of new
cuboid candidates along a spectrum between completely random and fully greedy choices.

\subsubsection{Exact Algorithms}
As the \gls{CLP} is a NP-hard problem, retrieving optimal solutions is computationally
challenging. The main difficulty is to represent packing patterns and the constraints
introduced in chapter \ref{sec:clp_definition} in a mathematocal mode. Two main methods
exist for modeling the \gls{CLP} mathematically. Firstly, \gls{MIP}, which focuses on
integer variables defining the placement and all considered constraints. This method is
often adapted by a Branch\&Price or Branch\&Cut approach, which helps to reduce the
search space and improve the solution time. Secondly, \gls{CP} is a more flexible
approach by defining constraints that variables must satisfy. This methods is used, when
complex constraints for the \gls{CLP} are considered \footcite[cf.][p. 7--11]{tamke_branch-and-cut_2024}.
In general are exact methods not capable of solving large instances with practical relevance
in reasonable-computation time. However, they can be used to understand the structure
of optimal solutions to provide lower bounds and guidance for the development of future
heuristics \footcite[cf.][p.2]{tamke_branch-and-cut_2024}.