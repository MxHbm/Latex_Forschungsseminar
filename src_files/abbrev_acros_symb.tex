\makeglossaries % Glossar, Abkürzungsverzeichnis und Symbolverzeichnis erstellen
\glsenableentrycount % aktiviert \cgls, \cglspl, \cGls, \cGlspl, siehe https://tex.stackexchange.com/questions/98494/glossaries-dont-print-single-occurences/230664#230664

% Abkürzungen, die im Abkürzungsverzeichnis auftauchen und automatisch durch das glossaries-Paket sortiert werden
\newacronym[description={Container Loading Problem}]{CLP}{CLP}{Container Loading Problem}
\newacronym[description={Vehicle Routing Problem}]{VRP}{VRP}{Vehicle Routing Problem}
\newacronym[description={Capacitated Vehicle Routing Problem}]{CVRP}{CVRP}{Capacitated Vehicle Routing Problem}
\newacronym[description={Bin Packing Problem}]{BPP}{BPP}{Bin Packing Problem}
\newacronym[description={Knapsack Problem}]{KP}{KP}{Knapsack Problem}
\newacronym[description={Last-In-First-Out}]{LIFO}{LIFO}{Last-In-–First-Out}
\newacronym[description={Cutting and Packing}]{CaP}{C\&P}{Cutting and Packing}
\newacronym[description={Two-dimensional}]{2D}{2D}{two-dimensional}
\newacronym[description={Three-dimensional}]{3D}{3D}{three-dimensional}
\newacronym[description={Constraint Programming}]{CP}{CP}{Constraint Programming}
\newacronym[description={Mixed-Integer Programming}]{MIP}{MIP}{Mixed-Integer Programming}
\newacronym[description={Genetic Algorithm}]{GA}{GA}{Genetic Algorithm}
\newacronym[description={Tabu Search}]{TS}{TS}{Tabu Search}
\newacronym[description={Simulated Annealing}]{SA}{SA}{Simulated Annealing}
\newacronym[description={Greedy Randomized Adaptive Search Procedure}]{GRASP}{GRASP}{Greedy Randomized Adaptive Search Procedure}
\newacronym[description={Variable Neighborhood Search}]{VNS}{VNS}{Variable Neighborhood Search}
\newacronym[description={Pallet Loading Problem}]{PLP}{PLP}{Pallet Loading Problem}
\newacronym[description={Bin Packing}]{BP}{BP}{Bin Packing}
\newacronym[description={Machine Learning}]{ML}{ML}{Machine Learning}
\newacronym[description = {Artificial Neural Network}]{ANN}{ANN}{Artificial Neural Network}
\newacronym[description = {Support Vector Machine}]{SVM}{SVM}{Support Vector Machine}
\newacronym[description = {Random Forest}]{RF}{RF}{Random Forest}
\newacronym[description = Two-dimensional Loading Capacitated Vehicle Routing Problem]{2L-CVRP}{2L-CVRP}{Two-dimensional Loading Capacitated Vehicle Routing Problem}
\newacronym[description = Three-dimensional Loading Capacitated Vehicle Routing Problem]{3L-CVRP}{3L-CVRP}{Three-dimensional Loading Capacitated Vehicle Routing Problem}
\newacronym[description = Three-Dimensional Loading Vehicle Routing Problem with Time Windows]{3L-VRPTW}{3L-VRPTW}{Three-Dimensional Loading Vehicle Routing Problem with Time Windows}


% Abkürzungen, die nicht im Abkürzungsverzeichnis aufgeführt werden
%\newabbreviation{zB}{z.\,B.}{zum Beispiel}

% Do i need this? 
\glsaddall[types=abbreviation]

% Symbole die im Symbolverzeichniss erscheinen sollen
% Mit "F5" kompilieren oder "Tools -> Befehle -> makeglossaries" (F9) starten, um das Symbolverzeichnis zu aktualisieren
% \newsymb{<sort by>}{<name>}{<symbol>}{<unit>}
%
\newsymb{E}{Erwartungswert}{\mathbb{E}\left(\cdot\right)}{}
\newsymb{P}{Wahrscheinlichkeitsmaß}{\mathbb{P}\left(\cdot\right)}{}
\newsymb{V}{Varianz}{\mathbb{V}\left(\cdot\right)}{}
\newsymb{X}{Zufallsvariable}{X}{}
%
% Oder so verwenden und im Fließtext dann mit $\ExpValue$ arbeiten:
%\newcommand{\ExpValue}{\mathbb{E}\left(\cdot\right)}
%\newsymb{E}{Erwartungswert}{\ExpValue}{}
%
\glsaddall[types={symbols}] % Alle Symbole werden dem Symbolverzeichnis hinzugefügt