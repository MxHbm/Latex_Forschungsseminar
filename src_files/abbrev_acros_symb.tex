\makeglossaries % Glossar, Abkürzungsverzeichnis und Symbolverzeichnis erstellen
\glsenableentrycount % aktiviert \cgls, \cglspl, \cGls, \cGlspl, siehe https://tex.stackexchange.com/questions/98494/glossaries-dont-print-single-occurences/230664#230664

% Abkürzungen, die im Abkürzungsverzeichnis auftauchen und automatisch durch das glossaries-Paket sortiert werden
\newacronym{crm}{CRM}{Customer Relationship Management}
\newacronym{scm}{SCM}{Supply Chain Management}
\newacronym[longplural={klein- und mittelständige Unternehmen},user1={klein- und mittelständigen Unternehmens}, description=Klein- und mittelständiges Unternehmen]{kmu}{KMU}{klein- und mittelständiges Unternehmen} % description tag setzt das Erscheinungsbild des Textes im Abkürzungsverzeichnis
\newacronym[description=Meine Welt]{mw}{mW}{meine Welt}

% Abkürzungen, die nicht im Abkürzungsverzeichnis aufgeführt werden
\newabbreviation{zB}{z.\,B.}{zum Beispiel}

% Do i need this? 
\glsaddall[types=abbreviation]

% Symbole die im Symbolverzeichniss erscheinen sollen
% Mit "F5" kompilieren oder "Tools -> Befehle -> makeglossaries" (F9) starten, um das Symbolverzeichnis zu aktualisieren
% \newsymb{<sort by>}{<name>}{<symbol>}{<unit>}
%
\newsymb{E}{Erwartungswert}{\mathbb{E}\left(\cdot\right)}{}
\newsymb{P}{Wahrscheinlichkeitsmaß}{\mathbb{P}\left(\cdot\right)}{}
\newsymb{V}{Varianz}{\mathbb{V}\left(\cdot\right)}{}
\newsymb{X}{Zufallsvariable}{X}{} 
%
% Oder so verwenden und im Fließtext dann mit $\ExpValue$ arbeiten:
%\newcommand{\ExpValue}{\mathbb{E}\left(\cdot\right)}
%\newsymb{E}{Erwartungswert}{\ExpValue}{}
%
\glsaddall[types={symbols}] % Alle Symbole werden dem Symbolverzeichnis hinzugefügt