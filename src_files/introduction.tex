\chapter{Introduction}

The task of packing several items with different geometries and sizes into a container
is a common problem in logistics and supply chain management. It occurs in several
different situations, such as loading cargo ships, trucks, or airplanes,
as well as in warehouse management and packing of pallets or cardboards with
smaller items \parencite{bortfeldt_constraints_2013}.

The challenge of packing items into a container is to find an optimal arrangement or for
complex problems, a feasible arrangement of items in the container,
such that the packing is applicable \parencite{krebs_advanced_2021}.
The \gls{CLP} for multi-dimensional cases is a well-known NP-hard problem. Therefore
only few exact approaches exist and the majority of publications focuses on
heuristics to obtain good solutions. The packing is further complicated by the presence
of constraints, such as item orientation, stability, and load-bearing capacity.
These constraints portay a more realistic picture of the problems encountered in practice
\parencite{bortfeldt_constraints_2013}.

When the focus of generating feasible packings is more important than finding the best
packing solution, it is possible to predict the feasibility with a trained model instead
of computing the exact solution. This approach was used in the paper \cite{zhang_learning-based_2022}
and significantly reduced the computation time for the integrated 2D packing problem
of the \gls{VRP}. However, the underlying constraints and assumptions do not reflect
a realistic scenario. Therefore, this paper aims to explore and analyze various
constraint datasets used in the 3D \gls{CLP}, laying the groundwork for feasibility
prediction of 3D packings under realistic constraints.

The paper is structured as follows: In the next chapter, the \gls{CLP} is formally defined
and different constraints are discussed. Chapter 3 gives an overview of classical solution
approaches and their limitations. Chapter 4 motivates the use of feasibility prediction
and gives an overview of machine learning classifiers. Chapter 5 compares available datasets
and justifies the selection of the dataset used in this paper. The conclusion and outlook
are presented in Chapter 6.

