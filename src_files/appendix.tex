%############################# Anhang #################################
%\appendix
%\chapter{Mathematischer Anhang}
%\chapter{Programmcodes}

\chapter{Guten Tag}

Das ist ein Test, ob in der Include Umgebung etwas stehen muss?

\section*{Appendix: Overview of Solution Techniques in 3D Container Loading Problems}

The following overview summarizes the solution techniques for 3D container loading problems as presented in Zhao et al. (2013) \cite{zhao_comparative_2016}:

\subsection*{1. Constructive Heuristics}
\begin{itemize}
    \item \textbf{Wall-building heuristics:} Pack boxes into vertical walls, using rules or search to determine wall width and item sequence.
    \item \textbf{Layer-building heuristics:} Load items in horizontal layers, typically from bottom to top, optimizing layer construction and stacking.
    \item \textbf{Block-based packing:} Pre-combine boxes into larger blocks (homogeneous or heterogeneous) to simplify placement.
    \item \textbf{Maximal-space heuristics:} Identify and fill maximal free cuboids remaining in the container with the best-fitting items.
    \item \textbf{Corner point (extreme point) placement:} Use available corner positions (formed by placed boxes) to place the next item.
    \item \textbf{Spatial representation methods:} Model the container as a 3D occupancy matrix or height profile to manage space dynamically.
    \item \textbf{Specialized heuristics:} Address specific scenarios such as identical box packing or directed-graph-based placement.
\end{itemize}

\subsection*{2. Improvement Heuristics and Metaheuristics}
\begin{itemize}
    \item \textbf{Genetic Algorithms (GA):} Optimize box sequences and packing heuristics via evolutionary strategies.
    \item \textbf{Bees Algorithm:} Bio-inspired metaheuristic that uses local and global search to refine packing layouts.
    \item \textbf{Tabu Search (TS):} Local search using memory structures to avoid cycling and enhance neighborhood exploration.
    \item \textbf{Simulated Annealing (SA):} Probabilistic search to escape local optima by accepting worse solutions under a cooling schedule.
    \item \textbf{Guided Local Search (GLS):} Penalizes undesirable features (e.g., overlaps) to drive the search toward better layouts.
    \item \textbf{GRASP:} Repeated greedy randomized construction with local improvement.
    \item \textbf{Variable Neighborhood Search (VNS):} Systematic change of neighborhood structures to diversify the local search.
    \item \textbf{Repeated greedy / lookahead searches:} Apply greedy heuristics iteratively or use tree-based lookahead to select placements.
\end{itemize}

\subsection*{3. Exact Optimization Methods}
\begin{itemize}
    \item \textbf{Column-generation approaches:} Solve master ILP problems with dynamically generated packing patterns.
    \item \textbf{Mixed-integer programming (MIP):} Exact formulations for packing box positions and container assignments, often with stability and load-bearing constraints.
    \item \textbf{Heuristic-driven exact algorithms:} Use MIP solvers for specific subproblems (e.g., adjusting layouts to meet axle weight constraints).
    \item \textbf{Approximation algorithms:} Decompose the problem into knapsack problems (e.g., vertical stacking and floor arrangement).
\end{itemize}

\subsection*{4. Hybrid and Integrated Techniques}
\begin{itemize}
    \item \textbf{Hybrid Genetic Algorithms:} Combine GAs with constructive heuristics or local search methods.
    \item \textbf{Hybrid Tabu Search:} Use TS in combination with domain-specific heuristics for container filling.
    \item \textbf{Sequential Metaheuristic Combinations:} Run different metaheuristics in sequence (e.g., SA followed by TS).
    \item \textbf{Heuristic–Metaheuristic Integrations:} Combine fast heuristic placements with higher-level metaheuristics for global optimization.
\end{itemize}

This structured classification highlights the diversity and complexity of solution strategies used in addressing the 3D container loading problem.
