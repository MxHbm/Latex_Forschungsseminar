\chapter{Motivation for Feasibility Prediction}
\label{sec:motivation_feasibility_prediction}
In this section it will be elaborated which advantages can be be obtained, when
a classifier is used to support the \gls{CLP} decision process. Two papers will be
analyzed, which use predictive methods to accelerate the computation time. Before that,
a short introduction to classifiers will be conducted.

A classifier is a type of supervised \gls{ML} algorithm used to predict the
value of a categorical or binary output column (also known as the label) based on the
values of other columns, called features. Classifiers learn from a labeled dataset,
where the correct output values are known in advance, and then use this knowledge to
make predictions on new, unseen data. A classifier can be implemented using various
\gls{ML} models: ...









It is important to note that several machine learning approaches feature
online \gls{BP}, where the best placement of items, consecutively available, needs
to be found, instead of finding one overall placement for all items at once
(Offline) \footcite[cf.][p. 1]{ali_-line_2022}. As this work focuses on the offline
\gls{BP}, the online approaches are not further considered.

\subsubsection{Feasibility Prediction of the \gls{CLP} integrated in \gls{VRP}}
A practical application of the \gls{CLP} is the integration in the \gls{VRP}, where
a number of customers needs to be served with a set of items by a fleet of vehicles that have
to start from a depot and return. The goal is to minimize the total distance driven
by the vehicles. When considering multidimensional items, the NP-hard problem itself,
has the \gls{CLP} alteration, which leads to increased complexity, as every tour is representing
a \gls{CLP} itself \footcite[cf.][pp.1--2]{tamke_branch-and-cut_2024}.
In order to reduce the computational effort of checking the feasibility of the \gls{CLP}
of single tours, \citeauthor{zhang_learning-based_2022} trained a classifier, which predicts
the feasibility of the \gls{CLP} for a given tour. This classifier helped to significantly
reduce the computation time of the \gls{CLP} and had an accuracy of at most $94.1\%$.
However, some simplifications were made, they focused only on the ground area of the items
considering a \gls{2D} \gls{CLP} and no further constraints, as unloading, rotation,
or fragility, were considered \footcite[cf.][pp. 4, 14]{zhang_learning-based_2022}.

\subsubsection{Classifier for the \gls{PLP}}
Another use of a classifier is presented by \citeauthor{aylak_application_2021}, where the pallet
size is chosen among three possible dimensions based on the number and uniform size of the items.
The classifier was trained with the results of several packing patterns of real-life datasets
to optimize the utilization of the pallet. An improvement of $6.7\%$ and a significant reduction
of the computation time was obtained in comparison to the heuristic approach
\footcite[cf.][pp. 12--14]{aylak_application_2021}.


\section{Trade-off Between Computation Time and Decision Support}
\section{Use Cases: Pre-filtering and Early-stage Decisions}
\section{Feasibility Prediction: Methods and Approaches}
\section{Overview of Machine Learning Classifiers}
\section{Data Dependency and Bias}






As the variety of \gls{CLP}s is quite broad, this work focuses specifically on
the application to \gls{VRP} with integrated \gls{3D} container loading. The emphasis lies
on the verification of packing feasibility, as the total distance is minimized.