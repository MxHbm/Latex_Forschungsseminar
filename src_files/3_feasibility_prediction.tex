\chapter{Motivation for Feasibility Prediction}
\label{sec:motivation_feasibility_prediction}
This section explores how \gls{ML} algorithms can enhance \gls{CLP} solution strategies,
highlighting both benefits and limitations. Two papers will be
analyzed, which use predictive methods to accelerate the computation time. Before that,
a short introduction to classifiers will be conducted. It is important to note that
several machine learning approaches feature
online \gls{BPP}, where the best placement of single items, consecutively available,
needs to be found, instead of finding one overall placement for all items at once
(Offline) \footcite[cf.][p. 1]{ali_-line_2022}. As this work focuses on offline
\gls{BPP}, the online approaches are not further considered, and the focus is laying
primarily on \gls{ML} classifiers. These are supervised \gls{ML} algorithms used to predict the
value of a categorical or binary output column (also known as the label) based on the
values of other columns, called features. Classifiers learn from a labeled dataset,
where the correct output values are known in advance, and then use this knowledge to
make predictions on new, unseen data. The accuracy can be evaluated afterwards by comparing
the predicted labels with the actual labels \footcite[cf.][]{kotsiantis_supervised_2007}.
A exemplary training dataset is shown in table \ref{tab:classifier_label_data}.

\begin{table}[h!]
    \small
    \centering
    \begin{tabular}{@{}cccccc@{}}
        \toprule
        \textbf{Instance} & \textbf{Feature 1} & \textbf{Feature 2} & \dots  & \textbf{Feature n} & \textbf{Label} \\
        \midrule
        1                 & xxx                & x                  & \dots  & xx                 & True           \\
        2                 & xxx                & x                  & \dots  & xx                 & False          \\
        3                 & xxx                & x                  & \dots  & xx                 & True           \\
        \vdots            & \vdots             & \vdots             & \vdots & \vdots             & \vdots         \\
        \bottomrule
    \end{tabular}
    \caption[Exemplary training data for a classifier with known labels.]{Exemplary training data for a classifier with known labels. \footnotemark}
    \footnotetext{Adapted table from \textcite[p. 249]{kotsiantis_supervised_2007}.}
    \label{tab:classifier_label_data}
\end{table}


A classifier can be implemented using various \gls{ML} models such as \gls{LR},
\gls{ANN}, \gls{SVM}, or others. However, the most crucial aspect of any
\gls{ML} algorithm is the selection of data, particularly the choice of features and
the size of the training set, since many models can be easily preselected from available
libraries. The model attempts to learn correlations between the provided features
and the corresponding labels. If the features are poorly chosen, the model may fail
to capture the underlying patterns in the data. Additionally, if the training set
is too small, the model might not generalize well, ultimately lacking the ability
to accurately predict unseen data. Furthermore, it needs to be noted, that available
\gls{ML} classifier are by nature not superior to others, but can significantly outperform
others on specific application problem \footcite[cf.][pp. 250, 264]{kotsiantis_supervised_2007}.
As the \gls{CLP} is a NP-hard problem, it is interesting to evaluate the usage of classifiers
to complement exisiting algorithm.

\subsubsection{Feasibility Classifier of the \cgls{2L-CVRP}}
A practical application of the \gls{CLP} is the integration in the \gls{VRP}, where
a number of customers need to be served with a set of items by a fleet of vehicles that have
to start from a depot and return. The goal is to minimize the total distance driven
by the vehicles. When considering multidimensional items, the NP-hard problem itself,
increases in complexity, as every tour is representing a \gls{CLP} itself \footcite[cf.][pp. 1--2]{tamke_branch-and-cut_2024}.
In order to reduce the computational effort of checking the feasibility of the \gls{CLP}
of single tours, \citeauthor*{zhang_learning-based_2022} trained a \gls{ANN} binary classifier
predicting the feasibility of single tours. Before single tours are either discarded or accepted
as new solutions, the loading is checked with an exact algorithm. As many single tours need to be evaluated,
this classifier helped significantly to reduce the computation time of the \gls{CLP} and had
an accuracy of at most $94.1\%$. However, some simplifications were made, they focused only
on the ground area of the items tackling the \cgls{2L-CVRP} and no further constraints,
as unloading, rotation, or fragility (see Chapter~\ref{sec:clp_definition}) were considered.
The classifier was trained with a dataset of $50.000$ tours obtained by the Column Generation
algorithm of the \cgls{2L-CVRP} containing 17 hand-crafted features capturing geometric
and spatial characteristics of the packing problem. These include the ratio of the total item area
to the container floor area, as well as the mean, standard deviation, minimum, and maximum of
the following four indicators:
\begin{itemize}
    \item[1.] the ratio of item width to item height,
    \item[2.] the ratio of item width to the container width,
    \item[3.] the ratio of item height to the container height,
    \item[4.] the ratio of each item’s area to the total container area.
\end{itemize}
The \gls{ANN} model was trained in many epochs
using the \textit{Mini-Batch Gradient Descent Algorithm}. By integrating the classifier into the
exisiting Branch\&Price algorithm, the CPU time was reduced by $54.12\%$ and the frequency of
calling the exact feasibility checker by $87.22\%$. However, the objective values are not significantly
lower and the authors assume that the prediction accuracy does not influence the solution quality
to a high extent, as the objectve values obtained by the \gls{LR} model are indifferent \footcite[cf.][pp. 4, 9--15]{zhang_learning-based_2022}

\subsubsection{Pallet Size Classifier for the \gls{PLP}}
Another use case for \gls{ML} in packing problems is presented by \citeauthor*{aylak_application_2021},
who focus on selecting the optimal pallet size in the context of the \gls{PLP}. Here a number of fixed items
need to be placed on paletts with fixed weight, length and height dimensions, optimizing the volume utilization
generating subsequently stability and minimizing the number of pallets needed. Based on real-life data
three candidate pallet sizes were considered and the best packing pattern must be found for each box
configuration. Therefore, multiple packing heuristics were applied to
generate feasible packing patterns and identify the best-performing size. These could be used as labels for several
\gls{ML} models, which were trained to predict the optimal pallet size based on four input features: the box
dimensions \{x,y,z\} and the demand quantity. Compared to the purely heuristic approach, the classifier-based
selection achieved a volume utilization improvement of $6.7\%$ and significantly reduced computation time.
However, the study did not consider additional constraints such as weight limits, stacking rules, or stability.\footcite[cf.][pp. 12--14]{aylak_application_2021}

\parbreak

These two examples show that classifiers can be succesfully integrated in exisiting \gls{CLP}
algorithms to reduce the computation time and the overall complexity, when the classifying model
is well trained. However, the training of a classifier is a time-consuming process and the
practicability of trainig and integrating a classifier needs to be evaluated in every single
case thorougly. As both examples had few constraints the classifier could be trained with a quite
small number of features. The training of a classifier is demanding and the usage needs to be levaraged to reach a promising
cost-benefit ratio. A well-fitting application is the prediction of the feasibility of
packing a given set of cuboids into a container - representing a single tour of the \gls{3L-CVRP} - similar to the approach
presented in Chapter~\ref{sec:motivation_feasibility_prediction} by \citeauthor*{zhang_learning-based_2022}\footcite[cf.][]{zhang_learning-based_2022}.
Many routes need to be evaluated in the \gls{3L-CVRP} to find the set of routes, which minimizes the
total travelled distance by all vehicles. The packing of the requested items does not need to
fulfil any optimization criteria, but only needs to be feasible and to consider introduced \gls{CLP} constraints\footcite[cf.][]{tamke_branch-and-cut_2024}.
As discussed in Chapter~\ref{sec:classical_solution_approaches}, the \gls{CLP} is an NP-hard problem,
making it computationally demanding to verify the packing feasibility of individual routes.
A classifier can therefore enhance the performance of existing algorithms by accelerating the speed
of checking single routes for feasibility. The exact packing solution is computed only when a final
solution is found, or just before columns generated in Branch-and-Price or Branch-and-Cut algorithms
are discarded. This use case has the potential to significantly reduce the overall computation time. \footcite[cf.][pp. 9--11]{zhang_learning-based_2022}.
