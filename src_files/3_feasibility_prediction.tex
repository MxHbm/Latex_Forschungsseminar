\chapter{Motivation for Feasibility Prediction}
\label{sec:motivation_feasibility_prediction}
In this section it will be elaborated which advantages can be obtained, when
a classifier is used to support the \gls{CLP} decision process. Two papers will be
analyzed, which use predictive methods to accelerate the computation time. Before that,
a short introduction to classifiers will be conducted. It is important to note that
several machine learning approaches feature
online \gls{BP}, where the best placement of single items, consecutively available,
needs to be found, instead of finding one overall placement for all items at once
(Offline) \footcite[cf.][p. 1]{ali_-line_2022}. As this work focuses on offline
\gls{BP}, the online approaches are not further considered, and the focus is laying
primarily on \gls{ML} classifiers. These are supervised \gls{ML} algorithm used to predict the
value of a categorical or binary output column (also known as the label) based on the
values of other columns, called features. Classifiers learn from a labeled dataset,
where the correct output values are known in advance, and then use this knowledge to
make predictions on new, unseen data. The accuracy can be evaluated afterwards by comparing
the predicted labels with the actual labels \footcite[cf.][]{kotsiantis_supervised_2007}.
A exemplary training dataset is shown in table \ref{tab:classifier_label_data}.

\begin{table}[h!]
    \small
    \centering
    \begin{tabular}{@{}cccccc@{}}
        \toprule
        \textbf{Instance} & \textbf{Feature 1} & \textbf{Feature 2} & \dots  & \textbf{Feature n} & \textbf{Label} \\
        \midrule
        1                 & xxx                & x                  & \dots  & xx                 & True           \\
        2                 & xxx                & x                  & \dots  & xx                 & False          \\
        3                 & xxx                & x                  & \dots  & xx                 & True           \\
        \vdots            & \vdots             & \vdots             & \vdots & \vdots             & \vdots         \\
        \bottomrule
    \end{tabular}
    \caption[Exemplary train data for a classifier with known labels.]{Exemplary train data for a classifier with known labels. \footnotemark}
    \footnotetext{Adapted table from \cite[p. 249]{kotsiantis_supervised_2007}.}
    \label{tab:classifier_label_data}
\end{table}


A classifier can be implemented using various \gls{ML} models such as decision tree,
\gls{ANN}, \gls{SVM}, \gls{RF}, or others. However, the most crucial aspect of any
\gls{ML} algorithm is the selection of data — particularly the choice of features and
the size of the training set — since many models can be easily preselected from available
libraries. The model attempts to learn correlations between the provided features
and the corresponding labels. If the features are poorly chosen, the model may fail
to capture the underlying patterns in the data. Additionally, if the training set
is too small, the model might not generalize well, ultimately lacking the ability
to accurately predict unseen data. Furthermore, it needs to be noted, that available
\gls{ML} classifier are by nature not superior to others, but can significantly outperform
others on specific application problem \footcite[cf.][pp. 250, 264]{kotsiantis_supervised_2007}.

\subsubsection{Feasibility Classifier of the \cgls{2L-CVRP}}
A practical application of the \gls{CLP} is the integration in the \gls{VRP}, where
a number of customers needs to be served with a set of items by a fleet of vehicles that have
to start from a depot and return. The goal is to minimize the total distance driven
by the vehicles. When considering multidimensional items, the NP-hard problem itself,
increases in complexity, as every tour is representing a \gls{CLP} itself \footcite[cf.][pp. 1--2]{tamke_branch-and-cut_2024}.
In order to reduce the computational effort of checking the feasibility of the \gls{CLP}
of single tours, \citeauthor*{zhang_learning-based_2022} trained a \gls{ANN} classifier
predicting this feasibility. This classifier helped to significantly
reduce the computation time of the \gls{CLP} and had an accuracy of at most $94.1\%$.
However, some simplifications were made, they focused only on the ground area of the items
tackling the \cgls{2L-CVRP} and no further constraints, as unloading, rotation,
or fragility, were considered \footcite[cf.][pp. 4, 14]{zhang_learning-based_2022}.

\subsubsection{Pallet Size Classifier for the \gls{PLP}}
Another use of a classifier is presented by \citeauthor*{aylak_application_2021}, where the pallet
size is chosen among three possible dimensions based on the number and uniform size of the items.
The classifier was trained with the results of several packing patterns of real-life datasets
to optimize the utilization of the pallet. An improvement of $6.7\%$ and a significant reduction
of the computation time was obtained in comparison to the heuristic approach
\footcite[cf.][pp. 12--14]{aylak_application_2021}.\parbreak

These two examples show that classifiers can be succesfully integrated in exisiting \gls{CLP}
algorithms to reduce the computation time and the overall complexity, when the classifying model
is well trained. However, the training of a classifier is a time-consuming process and the
practicability of trainig and integrating a classifier needs to be evaluated in every single
case thorougly.
