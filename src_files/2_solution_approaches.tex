\chapter{Classical Solution Approaches}
\label{sec:classical_solution_approaches}
The \gls{CLP} is a NP-hard combinatorial problem. \footcite[cf.][p. 11]{bortfeldt_constraints_2013}
Consequently, heuristics and metaheuristics were the dominating tools
in the early stages of this research field.\footcite[cf.][]{pisinger_heuristics_2002} The variety of methods
and their solution quality for solving the 2D \gls{CLP} developed in recent years. \footcite[cf.][p. 23]{iori_exact_2021}
The variety of 3D algorithms, especially for exact methods, is in comparison limited, as
\citeauthor*{zhao_comparative_2016} elaborated in their survey. \footcite[cf.][]{zhao_comparative_2016}
The following Figure~\ref{fig:solution_methods_overview} presents an overview of the solution methods
presented in the comparative review about 3D \gls{CLP}, divided in three categories constructive heuristics, metaheuristics
and exact methods. Hybrid methods and improvement heuristics are not explicitly shown in the figure,
as the former combines components from multiple categories, while the latter is subsumed within
metaheuristics. The depicted solution methods are described in the following.

\begin{figure}[htbp]
    \centering
    \resizebox{0.55\linewidth}{!}
    {
        \begin{tikzpicture}[grow cyclic, text width=2.7cm, align=flush center,
                level 1/.style={level distance=3.5cm,sibling angle=120},
                level 2/.style={level distance=3.5cm,sibling angle=55}
            ]

            \node {Solution Techniques \gls{CLP}}
            child { node {Exact \\ Methods}
                    child { node {\Gls{CP}}}
                    child { node {\Gls{MIP}}}
                    child { node {Branch-and-\\Cut}}
                    child { node {Branch-and-\\Price}}
                }
            child { node {Metaheuristics}
                    child { node[text width=3cm] {\Gls{GRASP}}}
                    child { node {\Gls{SA}}}
                    child { node {\Gls{TS}}}
                    child { node {\Gls{GA}}}
                }
            child { node {Constructive Heuristics}
                    child { node {Wall Building}}
                    child { node {Layer Building}}
                    child { node {Block based packing}}
                    child { node {Tower based packing}}
                }
            ;

        \end{tikzpicture}
    }
    \caption[Overview of solution methods for the CLP.]{Overview of solution methods for the CLP.}
    \label{fig:solution_methods_overview}
\end{figure}



\subsubsection{Constructive Heuristics}
For obtaining an initial feasible solution, two primary strategies are commonly
used depending on the heterogeneity of the items. In cases where the items are weakly homogeneous,
the problem dimensionality is reduced from 3D to 2D by constructing either
\textit{walls} or \textit{layers} of similarly sized items. These structures fill one
dimension of the container, typically either length and height or width and length, thereby
simplifying the remaining problem space. Moreover, when the layers or walls have
equal dimensions approximately, horizontal and vertical stability constraints are met.
Beyond these two dominant approaches, several adaptations exist to address item heterogeneity.
For example, \citeauthor*{gehring_genetic_1997} propose the construction of item
\textit{towers}, in which boxes are stacked on top of other items so that the base of the box is covered completly
by the box below it.
This effectively reduces the dimensions from 3D to a 2D floor space arrangement,
similar to the \gls{PLP}. \footcite[cf.][pp. 402--406]{gehring_genetic_1997}
Another method involves forming \textit{blocks} composed of identically shaped items.
These blocks are treated as single entities
during packing, significantly reducing the number of elements to be handled and thus
lowering overall problem complexity. Even though constructive heuristics are quite simple,
they are still widely used because of their simplicity and efficiency to obtain fast solutions.
\footcite[cf.][pp. 11--13]{tamke_branch-and-cut_2024}

\subsubsection{Metaheuristics}
Once an initial solution is found, metaheuristics or improvement heuristics can be applied
to improve it. Therefore a solution representation, such as a permutation of items, is required
to conduct changes of the current solution. \gls{GA}s were
used to either improve the walls, towers or layers found by the constructive heuristics
or by improving the quality of the permutation of all items. \footcite[cf.][]{gehring_genetic_1997}
\gls{TS} is a widely used approach for \gls{BPP} and \gls{CLP}. It is based on
the idea of iteratively improving a solution by exploring its neighborhood and simultaneously
storing certain moves or complete solutions in a tabu list to avoid back-cycling to
previous solutions, feasible or infeasible ones. \footcite[cf.][pp. 344--345]{gendreau_tabu_2006} \gls{SA} has rarely been used as a
standalone metaheuristic in the context of \gls{CLP}. Instead, it is often combined
with other approaches to leverage its cooling schedule, which allows the acceptance of
worse solutions at higher temperatures. This helps the algorithm to escape local optima
early in the search and gradually transition into a more focused intensification phase as
the temperature decreases. \gls{GRASP} has the advantage of controlling the selection of new
cuboid candidates along a spectrum between completely random and full greedy choices. \footcite[cf.][]{moura_grasp_2005}

\subsubsection{Exact Algorithms}
Retrieving optimal solutions for the \gls{CLP} is computationally challenging in comparison
to finding good solutions with heuristics. The main difficulty is to represent packing
patterns and the constraints introduced in Chapter~\ref{sec:clp_definition} in a mathematical way.
Two prominent methods exist for modeling the \gls{CLP}. The first one is \gls{MIP}, which can be
formulated in two primary ways. One approach defines each possible packing pattern as
a variable. \footcite[cf.][pp. 29--30]{zhu_prototype_2012} The second approach models
the placement of items using discrete coordinate variables. \footcite[cf.][pp. 4--8]{moura_integrated_2009}
Both formulations can benefit from enhancements such as Branch\&Price or Branch\&Cut
methods, which reduce the search space and improve solution time.
The second main method is \gls{CP}, which offers a flexible alternative for handling
complex constraints, where the focus is to find feasible solutions primarily and
not fulfilling an optimization criterion directly. \footcites[cf.][pp. 5--8]{kucuk_constraint_2022}[cf.][pp. 7--11]{tamke_branch-and-cut_2024} \citeauthor*{iori_exact_2021} states, that
\gls{CP} improved the results of 2D \gls{CLP} problems significantly and is a promising
field of future research.\footcite[cf.][p. 23]{iori_exact_2021}

\parbreak

In general exact methods are not capable of solving large instances with practical relevance
in reasonable computation time. However, they can be used to understand the structure
of optimal solutions to provide lower bounds and guidance for the development of future
heuristics. \footcite[cf.][p. 2]{tamke_branch-and-cut_2024} A possible approach in improving
the performance of exact algorithms could be the use of speed-ups, as \gls{ML} models, which
help identifying quickly good solutions and thereby reducing the need for solving instances
exactly in an iterative procedure, which is further elaborated in the next chapter.