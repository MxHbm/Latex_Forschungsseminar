\chapter{Conclusion and Outlook}
\label{sec:conclusion}

The variety of \gls{CLP} applications in logistics is wide, and the integration of \gls{CLP} in
\gls{VRP} algorithms is a daily practical use case. However as shown in the chapters are these problems
NP-hard and it is computationally expensive to find optimal solutions. The usage of \gls{ML} classifiers
can improve the solution quality and computation time. However, is tot complex top train a calssier, which
is able to label the \gls{3L-CVRP} with practical constraints as features to train the classifer need
to represent the numerical values. A good starting point is the dataset \krebsADataSetText, which includes
many different measurements and constraints to be used in a classifier. To train a \gls{3L-CVRP} classifier
with practical constraints, the following further steps need to be conducted. A train dataset needs to be created,
which exists of single tours labeled as infeasible or feasible beforehand with an exact \gls{CP} approach.
Afterwards in many iterations and epochs the features needed to predcit the label with a high accuracy need to be
evaluated and it can be checked, hwo challenging each single constraint is to include as feature. When
the accuracy reaches a satisfying level, the classifer can be implemented in a overall \gls{3L-CVRP} algorithm.
The classifer cna be used to predict the feasibiltiy and the algoritms with and without the classifier. Afterwards
more instances can be tested with the classifier to see, if also other problem instances presented in Chapter 5
can be succesfully predicted with the classifier and consequently benchmarks for these datasets can be created created.