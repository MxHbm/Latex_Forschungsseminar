\chapter{Conclusion and Outlook}
\label{sec:conclusion}
\begin{comment}
The variety of \gls{CLP} applications in logistics is wide, and the urge to consider practical constraints
and realistic datasets is high. For practitioners, heuristics are the first choice as good solutions
can be retrieved in satisfactory time. Exact solutions help to understand the structure of optimal solutions
and to evalutate and enhance existing heuristics. To accelerate exact methods different \gls{ML} enhancements
were presented and the \gls{3L-CVRP} was identified as a promising future use case for \gls{ML} classifiers.
The importance of realistic constraints of the \gls{CLP} were higlighted and the challenges faced when
training the classifier with such constraints were discussed. A dataset from \citeauthor*{krebs_advanced_2021} was identified
was selected for future training of a classifier. To train a \gls{3L-CVRP} classifier that incorporates practical constraints, several next steps are
required. First, a training dataset must be created, consisting of individual tours that are pre-labeled
as feasible or infeasible using an exact \gls{CP} approach. In subsequent training iterations and epochs,
relevant features for accurately predicting feasibility must be evaluated. This includes assessing how
challenging it is to encode each individual constraint as a feature. Once the classifier achieves a
satisfactory level of accuracy, it can be integrated into a complete \gls{3L-CVRP} algorithm. The
classifier will then be used to predict the feasibility of solutions, enabling a comparison between
algorithmic performance with and without the classifier. Finally, the model can be tested on additional
problem instances—such as those introduced in Chapter 5—to evaluate its generalization capabilities.
This will allow the creation of meaningful benchmarks across multiple datasets.
\end{comment}
The wide range of \gls{CLP} applications in logistics underscores the growing need to incorporate practical
constraints and realistic datasets. While heuristics remain the preferred choice for practitioners due to
their efficiency in producing good solutions within acceptable time frames, exact methods play a crucial
role in understanding the structure of optimal solutions and benchmarking heuristics.
To enhance exact approaches, various \gls{ML}-based examples have been presented, and the \gls{3L-CVRP}
has emerged as a promising candidate for the application of machine learning classifiers. This work
emphasized the importance of realistic constraints within the \gls{CLP} context and explored the challenges
of training classifiers under such conditions. A dataset from \citeauthor*{krebs_advanced_2021} was
identified as a suitable foundation for developing a classifier. The next steps involve creating a labeled
training dataset, where individual tours are classified as feasible or infeasible using an exact \gls{CP}
method. Through iterative training and feature evaluation, the goal is to determine which constraints can
be effectively represented and how they impact classification performance. Once a satisfactory level of
accuracy is achieved, the classifier can be integrated into a full \gls{3L-CVRP} algorithm to predict
feasibility and improve decision-making. Finally, the classifier will be tested on additional problem
instances—such as those presented in Chapter 5—to assess its generalizability. This will support the
development of meaningful benchmarks across diverse datasets and contribute to the advancement of
constraint-aware learning in combinatorial logistics problems.