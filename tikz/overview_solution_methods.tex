\begin{figure}[htbp]
    \centering
    \resizebox{0.6\linewidth}{!}
    {
        \begin{tikzpicture}[grow cyclic, text width=2.7cm, align=flush center,
                level 1/.style={level distance=3.5cm,sibling angle=120},
                level 2/.style={level distance=3.5cm,sibling angle=45}
            ]

            \node {Solution Techniques \gls{3D} \gls{CLP}}
            child { node {Exact Methods}
                    child { node {\gls{CP}}}
                    child { node {\gls{MIP}}}
                    child { node {Branch \& \\Cut}}
                    child { node {Branch \& Price}}
                    child { node {Approximation Algorithms}}
                }
            child { node {Metaheuristics}
                    child { node {\gls{GA}}}
                    child { node {\gls{TS}}}
                    child { node {\gls{SA}}}
                    child { node {Neighborhood Intensification}}
                    child { node {Bees Algorithm}}
                }
            child { node {Constructive Heuristics}
                    child { node {Wall Building}}
                    child { node {Layer Building}}
                    child { node {Block based packing}}
                    child { node {Tower based packing}}
                }
            ;

        \end{tikzpicture}
    }
    \caption{Overview of solution methods for the \gls{3D} \gls{CLP}\footnotemark}
    \footnotetext{Own figure based on overview of \cite{zhao_comparative_2016}}
    \label{fig:solution_methods_overview}
\end{figure}


%Wall Building
\begin{comment}
Wall-building heuristics: These fill the container by packing a series of vertical walls (sections of boxes) sequentially along the depth​
EPRINTS.SOTON.AC.UK
. A classic example is the two-stage algorithm of George & Robinson (1980) which builds walls using ranking rules based on box dimensions and quantities​
EPRINTS.SOTON.AC.UK
. Variations include Bischoff & Marriott’s (1990) exhaustive comparison of 14 wall-building heuristics (with different ranking and filling rules) and a combined hybrid approach running all 14 to choose the best result​
EPRINTS.SOTON.AC.UK
. Moura & Oliveira (2005) modify the George–Robinson method to adjust how new empty spaces are generated (restricting new wall width for stability)​
EPRINTS.SOTON.AC.UK
. Other extensions use tree search to decide wall dimensions instead of fixed rules (Chien & Wu 1998; Pisinger 2002)​
EPRINTS.SOTON.AC.UK
, handle multi-destination loads by sequentially packing by destination (Bischoff & Ratcliff 1995)​
EPRINTS.SOTON.AC.UK
, or employ a “layer-determining box (LDB)” to define wall thickness (Gehring et al. 1990) and allow remainder spaces to form new walls​
EPRINTS.SOTON.AC.UK
. The LDB concept was extended to create blocks of multiple walls (Davies & Bischoff 1999)​
EPRINTS.SOTON.AC.UK
. Some algorithms integrate wall-building into metaheuristics, e.g. using the LDB heuristic within a genetic algorithm​
EPRINTS.SOTON.AC.UK
or a tabu search​
EPRINTS.SOTON.AC.UK
.
\end{comment}